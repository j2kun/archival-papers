% vim: textwidth=80
\documentclass[10pt]{article}
\usepackage{fullpage}

%\usepackage[utf8]{inputenc}
\usepackage{amsmath}
\usepackage{amsthm}
\usepackage{graphicx}
\usepackage{caption}
\usepackage{subcaption}
\usepackage{amssymb}
\usepackage{amsfonts}

\title{On Coloring Resilient Graphs}
\author{Jeremy Kun \qquad Lev Reyzin\\ \\
Department of Mathematics, Statistics, and Computer Science\\
University of Illinois at Chicago\\
Chicago, IL 60607\\
\texttt{\{jkun2,lreyzin\}@math.uic.edu}
}

\newtheorem{thm}{Theorem}
\newtheorem{cor}{Corollary}
\newtheorem{defn}{Definition}
\newtheorem{propn}{Proposition}
\newtheorem{obs}{Observation}
\newtheorem*{prob}{Problem}


\date{}

%\institute{Department of Mathematics, Statistics, and Computer Science\\
%University of Illinois at Chicago\\
%\texttt{\{jkun2,lreyzin\}@math.uic.edu}}

\begin{document}
\maketitle

\begin{abstract} 
We introduce a new notion of resilience for constraint satisfaction problems,
with the goal of more precisely determining the boundary between NP-hardness
and the existence of efficient algorithms for resilient instances.  In
particular, we study $r$-resiliently $k$-colorable graphs, which are those
$k$-colorable graphs that remain $k$-colorable even after the addition of any
$r$ new edges.  We prove lower bounds on the NP-hardness of coloring
resiliently colorable graphs, and provide an algorithm that colors sufficiently
resilient graphs. This notion of resilience suggests an array of open
questions for graph colorability and other combinatorial problems.
\end{abstract}

\section{Introduction and related work}

An important goal in studying NP-complete combinatorial problems is to find
precise boundaries between tractability and NP-hardness. This is often done by
adding constraints to the instances being considered until a polynomial time
algorithm is found.  For instance, while SAT is NP-hard, the restricted $2$-SAT
and XOR-SAT versions are decidable in polynomial time.  

In this paper we present a new angle for pushing the boundary of NP-hardness.
We informally define the resilience of a constraint-based combinatorial problem
and we focus on the case of resilient graph colorability. Roughly speaking, a
positive instance is resilient if it remains a positive instance up to the
addition of a constraint. For example, an instance $G$ of Hamiltonian circuit
would be ``$r$-resilient'' if $G$ has a Hamiltonian circuit, and $G$ minus any
$r$ edges \emph{still} has a Hamiltonian circuit. In the case of coloring, we
say a graph $G$ is $r$-resiliently $k$-colorable if $G$ is $k$-colorable and
will remain so even if any $r$ edges are added. One would imagine that
\emph{finding} an $r$-coloring in a very resilient positive instance would
become easier, as that instance is very ``far'' from being not colorable.  And
in general, one can pose the question: how resilient can instances be and have
the search problem still remain hard?\footnote{We focus on the search versions
of the problems because the decision version on resilient instances induces the
trivial ``yes'' answer.}

For most NP-hard problems, there are natural definitions of resiliency in our
framework.  For instance, resilient positive instances for optimization
problems over graphs can be defined as those that remain positive instances
even up to the addition or removal of any edge.  For satisfiability, we say a
resilient instance is one where variables can be ``fixed'' and the formula
remains satisfiable. In problems like set-cover, we could allow for the removal
of a given number of sets. Indeed, this can be seen as a general notion of
resilience applying to fixing constraints in 
constraint satisfaction problems (CSPs), which have an extensive
literature~\cite{Kumar92}.

In this
paper, however, we focus on graph coloring. Resilience is defined up to the addition of
edges, and we first show that this is an interesting notion: many famous, well
studied graphs exhibit strong resilience properties. Then, perhaps
surprisingly, we prove that $3$-coloring a $1$-resiliently 3-colorable graph is
NP-hard -- that is, it is hard to color a graph even when it is guaranteed to
remain $3$-colorable under the addition of any edge.  Briefly, our reduction
works by mapping positive instances of 3-SAT to 1-resiliently 3-colorable
graphs and negative instances to graphs of chromatic number at least 4. An
algorithm which can color 1-resiliently 3-colorable graphs can hence
distinguish between the two. On the other hand, we observe that $3$-resiliently
$3$-colorable graphs have polynomial-time coloring algorithms (leaving the case
of 3-coloring $2$-resiliently $3$-colorable graphs tantalizingly open). We also
show that efficient algorithms exist for $k$-coloring
$\binom{k}{2}$-resiliently $k$-colorable graphs for all $k$, and discuss the
implications of our lower bounds. 

This paper is organized as follows. In the next two subsections we review the
literature on other notions of resilience and on graph coloring. In
Section~\ref{sec:resilient-sat} we inspect a resilient version of 6-SAT which
lays the groundwork for our main theorem on 1-resilient 3-coloring. In
Section~\ref{sec:resilient-coloring-obs-bounds} we formally define the resilient
graph coloring problem and present preliminary upper and lower bounds. In
Section~\ref{sec:main-thm} we prove our main theorem, and in
Section~\ref{sec:open-problems} we discuss open problems.



\subsection{Related work on resilience}

There are related concepts of resilience in the literature. Perhaps the closest
in spirit is Bilu and Linial's notion of stability~\cite{BL12}.  Their notion
is restricted to problems over metric spaces and attempts to explain why
real-world instances are easy.  They argue that practical instances often
exhibit some degree of stability, which can make the problem easier.  Their
results on clustering stable instances have seen considerable interest and have
been substantially extended and improved~\cite{ABS10,BL12,Rey11}.  Their study
is also not limited to clustering -- for instance one can study TSP and other
optimization problems over metrics under the Bilu-Linial
assumption~\cite{MSSW11}.  
A related notion of stability by 
Ackerman and Ben-David~\cite{AckermanB09} for clustering 
yields efficient algorithms when the data lies in Euclidian space.

Our notion of resilience, on the other hand, is most natural in the case when
the optimization problem has natural constraints, which can be fixed or
modified.  Our primary goal is also different -- we seek to more finely
delineate the boundary between tractability and hardness in a systematic way
across problems.

Property testing can also be viewed as involving resilience. Roughly speaking
property testers are randomized algorithms which distinguish between
combinatorial structures that satisfy a property or are very far from satisfying
it. These algorithms are typically given access to a small sample depending on
$\varepsilon$ alone (and not the size of the input). For graph property testing,
as with resilience, the concept of being ``far'' from having a property involves
requiring the addition or removal of an arbitrary set of at most $\varepsilon
\binom{n}{2}$ edges from $G$. Our notion of resilience is different in that we
consider adding or removing a constant number of constraints. More importantly,
property testing is more concerned with query complexity than with
computational hardness, whereas we seek to relate the
complexity of search problems across varying degrees of resilience. 

\subsection{Previous work on coloring}

As the technical results of this paper are on graph colorability, we review
the relevant work.

A graph $G$ is $k$-colorable if there is an assignment of $k$ distinct colors
to the vertices of $G$ so that no edge is monochromatic. Determining whether
$G$ is $k$-colorable is a classic an NP-hard problem~\cite{Karp72}. Many 
attempts to simplify the problem, such as assuming planarity or bounded degree
of the graph in question, still result in NP-hardness~\cite{Dailey80}. A large
body of work surrounds positive and negative results for explicit families of
graphs. The list of families that are polynomial-time colorable includes
triangle-free planar graphs, perfect graphs and almost-perfect graphs, bounded
tree- and clique-width graphs, quadtrees, and various families of graphs
defined by the lack of an induced
subgraph~\cite{HMM10,Ko03,EBH99,Cai03,KKTW01}.

With little progress on coloring general graphs, research has naturally turned
to approximation. In approximating the chromatic number of a general graph, the
first results were of Garey and Johnson, giving a performance guarantee of
$O(n/\log n)$ colors~\cite{Johnson74} and proving that it is NP-hard to
approximate chromatic number to within a constant factor less than
two~\cite{GJ76}. Further work improved this bound by logarithmic factors
\cite{BR90,Ha93}. In terms of lower bounds, Zuckerman~\cite{Zu07} derandomized
the PCP-based results of H{\aa}stad~\cite{Ha99} to prove the best known
approximability lower-bound to date, $O(n^{1-\varepsilon})$.

There has been much recent interest in coloring graphs which are already known
to be colorable while minimizing the number of colors used. For a 3-colorable
graph, Widgerson gave an algorithm using at most $O(n^{1/2})$
colors~\cite{Wi83}, which Blum improved to $\tilde{O}(n^{3/8})$~\cite{Blum94}.
A line of research based on semidefinite programming and other combinatorial
improvements improved this bound still further to $\tilde{O}(n^{0.2072})$
\cite{Ka98,Blum97,Arora06,Chl09, KT12}. Despite the difficulties in improving
the constant in the exponent, and as suggested by Arora~\cite{Arora11}, there
is no evidence that coloring a 3-colorable graph with as few as $O(\log n)$
colors is hard.

On the other hand there are asymptotic and concrete lower bounds.
Khot~\cite{Khot01} proved that for sufficiently large $k$ it is NP-hard to
color a $k$-colorable graph with fewer than $k^{O(\log{k})}$ colors, and this
was recently improved by Huang to $2^{\sqrt[3]{k}}$~\cite{Huang13}. It is also
known that for every constant $h$ there exists a sufficiently large $k$ such
that coloring a $k$-colorable graph with $hk$ colors is NP-hard~\cite{DMR06}.
In the non-asymptotic case, Khanna, Linial, and Safra~\cite{KLS00} used the PCP
theorem to prove it is NP-hard to 4-color a 3-colorable graph, and more
generally to color a $k$ colorable graph with at most $k + 2\left \lfloor k/3
\right \rfloor - 1$ colors. Guruswami and Khanna later gave an explicit
reduction for $k=3$~\cite{GuKh2000}. Assuming a variant of Khot's 2-to-1
conjecture, Dinur et al. prove that distinguishing between chromatic number $K$
and $K'$ is hard for $3 \leq K < K'$~\cite{DMR06}.  Interestingly, this is the
best conditional lower bound we give in Section~\ref{sec:easy-bounds}, but it
does not to our knowledge imply Theorem~\ref{thm:3-1}.

Without large strides in approximate graph coloring, we need a new avenue to
approach the boundary between when coloring is difficult and when it is easy.
In this paper we consider the coloring problem for a general family of graphs
which we call \emph{resiliently colorable}, in the sense that adding edges does
not violate the given colorability assumption. 


\section{Warm up: resilient SAT}\label{sec:resilient-sat}

We begin by describing a resilient version of the $k$-satisfiability problem,
which is used in proving our main result for resilient coloring in Section~\ref{sec:main-thm}.
$k$-SAT also serves as a nice problem to use in introducing our more general notion of resilience.

\begin{prob}[resilient $k$-SAT]
A boolean formula $\varphi$ is called \emph{$r$-resilient} if it is satisfiable
and remains satisfiable if any set of $r$ variables are fixed. We call
\emph{$r$-resilient $k$-SAT} the problem of finding a satisfying assignment for
an $r$-resiliently satisfiable formula in $k$-CNF form. More generally, we call
$r$-resilient CNF-SAT the problem of finding a satisfying an assignment for an
$r$-resilient formula in CNF form.
\end{prob}

While it is unclear exactly which values of $k$ and $r$ make this problem
NP-hard, the following proposition is straightforward. We use it in
Section~\ref{sec:main-thm} as the starting point for our main reduction. 

\begin{propn}
1-resilient 6-SAT is NP-hard.
\end{propn}
\begin{proof}
We reduce from 3-SAT. Let $\mathbf{x} = (x_1, \dots, x_n)$ and suppose
$\varphi(\mathbf{x})$ is an instance of 3-SAT.  Construct $\psi =
\varphi(\mathbf{x}) \vee \varphi(\mathbf{y})$ for some choice of new variables
$\mathbf{y} = (y_1, \dots, y_n)$, and put $\psi$ into 6-CNF form (which can be done in $O(n^2)$
steps). The formula $\varphi$ is satisfiable if and only if $\psi$ is, and
$\psi$ is 1-resilient: if any of the $x_i$ are fixed, we may satisfy $\varphi$
using the $y_i$ and vice versa if any of the $y_i$ are fixed.  
\end{proof}

\begin{cor}
For all fixed $r \geq 0$, $r$-resilient CNF-SAT is NP-hard.
\end{cor}
\begin{proof}
The technique used in the previous proof can be extended to take a disjunction
of $r+1$ copies of $\varphi$, each with different variable sets. Call this new
formula $\psi$. Again $\varphi$ is satisfiable if and only if $\psi$ is, and we
can put $\psi$ into CNF form in polynomial time. By the pigeonhole principle if
$r$ variables are fixed then at least one disjunct has all of its variables
untouched, and we can use this set of variables to satisfy the whole formula.
\end{proof}

We now move on to graph coloring, the main subject of our study.

\section{Resilient graph coloring, observations, and preliminary bounds}
\label{sec:resilient-coloring-obs-bounds}
\subsection{Problem definition and remarks}

\begin{prob}[resilient coloring] 
A graph $G$ is called \emph{$r$-resiliently
$k$-colorable} if $G$ remains $k$-colorable under the addition of any set of
$r$ new edges.\footnote{It is also natural to consider the removal of edges as
well as their addition, but as the removal of edges can only decrease the
chromatic number, it is unlikely that this form of resilience makes coloring
easier.} 
\end{prob}

We argue that this notion is not trivial by showing the resilience properties
of some classic graphs.  These were determined by exhaustive computer search,
and are displayed in Figure~\ref{fig:famous-graphs}.  The Petersen graph is
2-resiliently 3-colorable. The D{\"u}rer graph is 1-resiliently 3-colorable
(but not 2-resilient) and 4-resiliently 4-colorable (but not 5-resilient). The
Gr{\"o}tzsch graph is 4-resiliently 4-colorable (but not 5-resilient). The
Chv{\'a}tal graph is 3-resiliently 4-colorable (but not 4-resilient).

\begin{figure}
\centering
\scalebox{0.45}{\includegraphics{famous-graphs.png}}
\caption{From left to right: the Petersen graph, 2-resiliently 3-colorable;
the D{\"u}rer graph, 4-resiliently 4-colorable; the Gr{\"o}tzsch graph,
4-resiliently 4-colorable; and the Chv{\'a}tal graph, 3-resiliently
4-colorable.} 
\label{fig:famous-graphs}
\end{figure}

There are a few interesting constructions to build intuition about resilient
graphs. First, it is clear that every $k$-colorable graph is 1-resiliently
$(k+1)$-colorable (just add one new color for the additional edge), but for all
$k > 2$ there exist $k$-colorable graphs which are not 2-resiliently
$(k+1)$-colorable. Simply remove two disjoint edges from the complete graph on
$k+2$ vertices. A slight generalization of this argument (removing a maximal
number of pairwise-disjoint edges from complete graphs) provides examples of
graphs which are $\left \lfloor (k+1)/2\right \rfloor$-colorable but not $\left
\lfloor (k+1)/2 \right \rfloor$-resiliently $k$-colorable for all $k \geq 3$.
On the other hand, every $\left \lfloor (k+1)/2\right \rfloor$-colorable graph
is $(\left \lfloor (k+1)/2 \right \rfloor-1)$-resiliently $k$-colorable. This
follows from the fact that $r$-resiliently $k$-colorable graphs are
$(r+m)$-resiliently $(k+m)$-colorable for all $m \geq 0$ (we have one new color
for each added edge). 

One might expect high resilience in a $k$-colorable graph to be a strong enough
property as to reduce the number of colors required to color it. While we
expect this to be the case for super-linear resilience, there are trivial
(and non-vacuous) examples of $(k-1)$-resiliently $k$-colorable graphs which are
$k$-chromatic. For instance, add an isolated vertex to the complete graph on
$k$ vertices. The Gr{\"o}tzsch and D{\"u}rer graphs also give nice examples
of $k$-resiliently $k$-colorable graphs for $k=4$. 

\subsection{Observations}

We are primarily interested in the complexity of coloring resilient graphs, and
so we pose the question: for which values of $k,r$ does the task of
$k$-coloring an $r$-resiliently $k$-colorable graph admit an efficient
algorithm? The following observations aid us in classifying such pairs. Using
these observations and the propositions in the following section, we fill in
the known complexities (including Theorem~\ref{thm:3-1}) in
Figure~\ref{fig:classification}.

\begin{obs}\label{obs:horizontal}
An $r$-resiliently $k$-colorable graph is $r'$-resiliently $k$-colorable for
any $r' \leq r$. Hence, if $k$-coloring is in P for $r$-resiliently
$k$-colorable graphs, then it is for $s$-resiliently $k$-colorable graphs for
all $s \geq r$.  Conversely, if $k$-coloring in NP-hard for $r$-resiliently
$k$-colorable graphs, then it is for $s$-resiliently $k$-colorable graphs for
all $s \leq r$. 
\end{obs}

This observation manifests itself in Figure~\ref{fig:classification} by the rule
that if a cell is in P, so are all of the cells to its right. Similarly, if a
cell is NP-hard, so are all of the cells to its left. We have a similar
observation for vertical implications. 

\begin{obs}\label{obs:vertical}
If $k$-coloring is in P for $r$-resiliently $k$-colorable graphs, then
$k'$-coloring $r$-resiliently $k'$-colorable graphs is in P for all $k' \leq
k$. Similarly, if $k$-coloring is in NP-hard for $r$-resiliently $k$-colorable
graphs, then $k'$-coloring is NP-hard for $r$-resiliently $k'$-colorable graphs
for all $k' \geq k$.
\end{obs}
\begin{proof}
If $G$ is $r$-resiliently $k$-colorable, then we construct $G'$ by adding a
new vertex $v$ with complete incidence to $G$. Then $G'$ is $r$-resiliently
$(k+1)$-colorable, and an algorithm to color $G'$ can be used to color $G$.
\end{proof}

Observation~\ref{obs:vertical} manifests itself in
Figure~\ref{fig:classification} by the rule that if a cell is in P, so are all
of the cells above it; if a cell is NP-hard, so are the cells below it.

More generally, we have the following observation which allows us to apply known
results to our problem.

\begin{obs}\label{obs:function-bound}
If it is NP-hard to $f(k)$-color a $k$-colorable graph, then it is NP-hard to
$f(k)$-color an $(f(k)-k)$-resiliently $f(k)$-colorable graph.
\end{obs}

This observation is used in Propositions~\ref{propn:4-1}
and~\ref{propn:2-to-1-diagonal}, and follows from the fact that an
$r$-resiliently $k$-colorable graph is $(r+m)$-resiliently $(k+m)$-colorable
for all $m \geq 0$ (here $r = 0, m = f(k) - k$). 

\begin{figure}
\centering
\begin{subfigure}{.65\textwidth}
  \centering
  \scalebox{0.55}{\includegraphics{table.png}}
\end{subfigure}%
\begin{subfigure}{.35\textwidth}
  \centering
  \scalebox{0.5}{\includegraphics{zoomed-out.png}}
\end{subfigure}
\caption{The known classification of the complexity of $k$-coloring
$r$-resiliently $k$-colorable graphs. Left: the explicit classification for
small $k, r$, referencing the propositions and theorems in this paper. Right: a
zoomed-out view of the same table, with the additional NP-hard (black) region
added by Proposition~\ref{propn:asymptotic}.} 
\label{fig:classification} 
\end{figure}

\subsection{Upper and lower bounds} \label{sec:easy-bounds}
In this section we provide a simple upper bound on the complexity of coloring
resilient graphs, we apply known results to show that 4-coloring a
1-resiliently 4-colorable graph is NP-hard, and we give the conditional hardness
of $k$-coloring $(k-3)$-resiliently $k$-colorable graphs for all $k \geq 3$.
This last result follows from the work of Dinur, et al., and depends a variant
of Khot's 2-to-1 conjecture~\cite{DMR06}. Finally, applying the result of
Huang~\cite{Huang13}, we give an unconditional asymptotic lower bound. These
results, along with our main theorem, are displayed graphically in
Figure~\ref{fig:classification}.

To explain Figure~\ref{fig:classification} more explicitly,
Proposition~\ref{propn:k-choose-2-bound} gives an upper bound for $r=
\binom{k}{2}$, and Proposition~\ref{propn:4-1} gives hardness of the cell $(4,1)$
and its consequences. Proposition~\ref{propn:2-to-1-diagonal} provides the
conditional lower bound, and Theorem~\ref{thm:3-1} gives the hardness of the cell
$(3,1)$. Finally, Proposition~\ref{propn:asymptotic} provides the asymptotic
lower bound.

\begin{propn}\label{propn:k-choose-2-bound}
There is an efficient algorithm for $k$-coloring $\binom{k}{2}$-resiliently
$k$-colorable graphs.
\end{propn}
\begin{proof}
If $G$ is $\binom{k}{2}$-resiliently $k$-colorable, then no vertex may have
degree $ \geq k$. For if $v$ is such a vertex, one may add complete incidence to
any choice of $k$ vertices in the neighborhood of $v$ to get $K_{k+1}$. Finally,
graphs with bounded degree $k-1$ are greedily $k$-colorable. 
\end{proof}

\begin{propn}\label{propn:4-1}
The problem of 4-coloring a 1-resiliently 4-colorable graph is NP-hard.
\end{propn}
\begin{proof}
It is known that 4-coloring a 3-colorable graph is NP-hard, so we may apply
Observation~\ref{obs:function-bound}. Every 3-colorable graph $G$ is
1-resiliently 4-colorable, since if we are given a proper 3-coloring of $G$ we
may use the fourth color to properly color any new edge that is added. So an
algorithm $A$ which efficiently 4-colors 1-resiliently 4-colorable graphs can be
used to 4-color a 3-colorable graph.
\end{proof}

We call a problem \emph{2-to-1-hard} if it is NP-hard under the condition that
Khot's 2-to-1 conjecture holds.

\begin{propn}\label{propn:2-to-1-diagonal}
For all $k \geq 3$, it is 2-to-1-hard to $k$-color a $(k-3)$-resiliently
$k$-colorable graph. 
\end{propn}
\begin{proof}
As with Proposition~\ref{propn:4-1}, we apply Observation~\ref{obs:function-bound}
to the conditional fact that it is NP-hard to $k$-color a 3-colorable graph for
$k > 3$. Such graphs are $(k-3)$-resiliently $k$-colorable.
\end{proof}

We may apply Observation~\ref{obs:function-bound} once again to asymptotic
bounds, such as the lower bound of Huang~\cite{Huang13}.

\begin{propn}\label{propn:asymptotic}
For sufficiently large $k$ it is NP-hard to $2^{\sqrt[3]{k}}$-color an
$r$-resiliently $2^{\sqrt[3]{k}}$-colorable graph for $r < 2^{\sqrt[3]{k}} - k$.
\end{propn}

The only unexplained cell of Figure~\ref{fig:classification} is (3,1), which we
prove is NP-hard as our main theorem in the next section.

\section{NP-hardness of 1-resilient 3-colorability}\label{sec:main-thm}

We now prove a non-trivial concrete lower bound, giving the hardness of
$3$-coloring $1$-resiliently $3$-colorable graphs. 

\begin{thm}\label{thm:3-1}
It is NP-hard to 3-color a 1-resiliently 3-colorable graph.
\end{thm}
\begin{proof}
We reduce 1-resilient 3-coloring from 1-resilient 6-SAT. This reduction comes
in the form of a graph which is 3-colorable if and only if the 6-SAT instance is
satisfiable, and 1-resiliently 3-colorable when the 6-SAT instance is
1-resiliently satisfiable. We fix the three colors used in the discussion to
white, black, and gray.

We first describe the gadgets involved and prove their consistency (that the
6-SAT instance is satisfiable if and only if the graph is 3-colorable), and
then prove the construction is 1-resilient. Given a 6-CNF formula $\varphi = C_1
\wedge \dots \wedge C_m$ we construct a graph $G$ as follows. Start with a base
vertex $b$ which we may assume without loss of generality is always colored
gray. For each literal we construct a \emph{literal gadget} consisting of two
vertices both adjacent to $b$, as in Figure~\ref{fig:literal-gadget}. As such
the vertices in a literal gadget may only assume the colors white and black. A
variable is interpreted as true if and only if both vertices in the literal
gadget have the same color. We will abbreviate this by saying a literal is
\emph{colored true} or \emph{colored false}.

\begin{figure}[bht]
\centering
\scalebox{0.45}{\includegraphics{literal-gadget.png}}
\caption{The gadget for a literal. The two single-degree vertices represent a
single literal, and are interpreted as true if they have the same color. The
base vertex is always colored gray.}
\label{fig:literal-gadget}
\end{figure}

We connect two literal gadgets for $x, \overline{x}$ by a \emph{negation
gadget} in such a way that the gadget for $x$ is colored true if and only if
the gadget for $\overline{x}$ is colored false. The negation gadget is given in
Figure~\ref{fig:clause-and-negation-gadget}. In the diagram, the vertices
labeled 1 and 3 correspond to $x$, and those labeled 10 and 12 correspond to
$\overline{x}$. We start by showing that no proper coloring can exist if both
literal gadgets are colored true. If all four of these vertices are colored
white or all four are black, then vertices 6 and 7 must also have this color,
and so the coloring is not proper. If one pair is colored both white and the
other both black, then vertices 13 and 14 must be gray, and the coloring is
again not proper.  Next, we show that no proper coloring can exist if both
literal gadgets are colored false. First, if vertices 1 and 10 are white and
vertices 3 and 12 are black, then vertices 2 and 11 must be gray and the
coloring is not proper. If instead vertices 1 and 12 are white and vertices 3
and 10 black, then again vertices 13 and 14 must be gray. This covers all
possibilities up to symmetry.  It is also easy to see that whenever one literal
is colored true and the other false, one can extend it to a proper 3-coloring
of the whole gadget. 

\begin{figure}
\centering
\scalebox{0.52}{\includegraphics{clause-and-negation-gadget.png}}
\caption{Left: the gadget for a clause. Right: the negation gadget ensuring two
literals assume opposite truth values.} 
\label{fig:clause-and-negation-gadget}
\end{figure}

Now suppose we have a clause involving literals, without loss of generality
$x_1, \dots, x_6$. We construct the \emph{clause gadget} shown in
Figure~\ref{fig:clause-and-negation-gadget}, and claim that this gadget is
3-colorable if and only if at least one literal is colored true. Indeed, if the
literals are all colored false, then the vertices labeled 13 through 18 in the
diagram must be colored gray, and as a result the vertices 25, 26, 27
must be gray. This causes the central triangle to use only white and black,
and so it cannot be a proper coloring. On the other hand, if some literal is
colored true, we claim we can extend to a proper coloring of the whole gadget.
Suppose without loss of generality that the literal in question is $x_1$, and
that vertices 1 and 2 both are black. Then we see in
Figure~\ref{fig:clause-gadget-proof} how this extends to a proper coloring of
the entire gadget regardless of the truth assignments of the other literals (we
can always color their branches as if the literals were false). 

\begin{figure}
\centering
\scalebox{0.5}{\includegraphics{clause-gadget-proof.png}}
\caption{A valid coloring of the clause gadget when one variable (in this case $x_3$) is true.}
\label{fig:clause-gadget-proof}
\end{figure}

It remains to show that $G$ is 1-resiliently 3-colorable when $\varphi$ is
1-resiliently satisfiable. This is because the worst that a new edge can do is
fix the truth assignment (perhaps indirectly) of at most one literal. Since the
original formula $\varphi$ is 1-resiliently satisfiable, $G$ maintains
3-colorability.  Additionally, the gadgets and the representation of truth were
chosen in such a way as to provide flexibility with respect to the chosen colors
for each vertex, so many edges will have no effect on the colorability of $G$. 

First, one can verify that the gadgets themselves are 1-resiliently
3-colorable.\footnote{Again, these graphs are sufficiently small so as to admit
verification by computer search.} We will use this fact throughout the rest of
the proof. We break down the analysis into eight cases based on the endpoints of
the added edge. These cases are: within a single clause/negation/literal gadget,
between two distinct clause/negation/literal gadgets, between clause and
negation gadgets, and between negation and literal gadgets. We denote the added
edge by $e = (v,w)$ and call it \emph{good} if $G$ is still 3-colorable after
adding $e$. 

\emph{Literal Gadgets}. First, we argue that $e$ is good if it lies within or
across literal gadgets. Indeed, there is only one way to add an edge within a
literal gadget, and this has the effect of setting the literal to false. If $e$
lies across two gadgets then it has no effect: if $c$ is a proper coloring of
$G$ without $e$, then after adding $e$ either $c$ is still a proper coloring or
we can switch to a different representation of the truth value of $v$ or $w$ to
make $e$ properly colored (i.e. swap ``white white'' with ``black black,'' or
``white black'' with ``black white'' and recolor appropriately). 

\emph{Negation Gadgets}. Next we argue that $e$ is good if it involves a
negation gadget. Let $N$ be a negation gadget for the variable $x$. Indeed, by
1-resilience an edge within $N$ is good; $e$ only has a local effect within
negation gadgets, and it may result in fixing the truth value of $x$. Now
suppose $e$ has only one vertex $v$ in $N$. Figure~\ref{fig:coloring-negation}
shows two ways to color $N$, which together with reflections along the
horizontal axis of symmetry have the property that we may choose from at least
two colors for any vertex we wish. That is, if we are willing to fix the truth
value of $x$, then we may choose between one of two colors for $v$ so that $e$
is properly colored regardless of which color is adjacent to it.  

\begin{figure}
\centering
\scalebox{0.5}{\includegraphics{coloring-negations.png}}
\caption{Two distinct ways to color a negation gadget without changing the
truth values of the literals. Note that only one vertex (the rightmost center
vertex) cannot be given a different color by a suitable switch between the two
representations or a reflection of the graph across the horizontal axis of
symmetry. If the new edge involves this vertex, we must fix the truth value
appropriately.}
\label{fig:coloring-negation}
\end{figure}

\emph{Clause Gadgets}. It remains to inspect the case that $e$ lies within a
clause gadget or between two clause gadgets. As with the negation gadget, it
suffices to fix the truth value of one variable suitably so that one may choose
either of two colors for one end of the new edge. Although it is not always
necessary to fix a variable to accommodate the new edges, it is possible in all
cases.  Figure~\ref{fig:clause-clause-example} provides a detailed illustration
of one such case. Here, we focus on two branches of two separate clause
gadgets, and add the new edge $e = (v,w)$. The added edge has the following
effect: if $x$ is false, then neither $y$ nor $z$ may be used to satisfy $C_2$
(as $w$ cannot be gray). This is no stronger than requiring that either $x$ be
true or $y$ and $z$ both be false, i.e., we add the clause $x \vee
(\overline{y} \wedge \overline{z})$ to $\varphi$. Since this clause can be
satisfied by fixing a single variable ($x$ to true), and $\varphi$ is
1-resilient, we can still satisfy $\varphi$ and hence 3-color $G$. The other
cases are nearly identical, except that the added clause changes; fixing a
single variable suffices to satisfy the new clause. 

This proves that $G$ is 1-resilient when $\varphi$ is, and finishes the proof.
\end{proof}

We note that the clause gadget in the proof above 
comes from Kun~et~al.~\cite{KunPR13}, who employ this 
construction in analyzing a different graph coloring problem.


\begin{figure}
\centering
\scalebox{0.5}{\includegraphics{clause-clause-example.png}}
\caption{An example of an edge added between two clauses $C_1, C_2$.}
\label{fig:clause-clause-example}
\end{figure}



\section{Discussion and open problems} \label{sec:open-problems}

The notion of resilience introduced in this paper leaves many questions
unanswered, both specific problems about  graph coloring and more
general exploration of resilience in other combinatorial problems and CSPs. 

Regarding resilient graph coloring, our paper established the fact that
1-resilience doesn't affect the difficulty of graph coloring. However, the
question of 2-resilience is wide open, as is establishing linear lower bounds
without dependence on the 2-to-1 conjecture. There is also obvious room for
improvement in finding efficient algorithms for highly-resilient instances,
closing the gap between NP-hardness and tractability.

Additionally, we only scratched the surface of questions about resilient
satisfiability. For instance, while 3-resilient 3-SAT is vacuously trivial, it
is unclear whether 1- or 2-resilience is strong enough to make finding a
satisfying assignment of a 3-SAT instance easy. We conjecture that 1-resilience
is NP-hard and 2-resilience is tractable.

On the general side, there is a wealth of NP-complete problems that our
framework could be applied to, including Hamiltonian circuit, set cover,
3D-matching, 0-1 linear programming, and many others. Each presents its own
boundary between NP-hardness and tractability, and there are undoubtedly
interesting relationships across problems, as we showed with resilient
satisfiability and resilient coloring.

\subsection*{Acknowledgments}
We thank Shai Ben-David for helpful discussions.


\bibliographystyle{plain}
\bibliography{paper}


\end{document}
